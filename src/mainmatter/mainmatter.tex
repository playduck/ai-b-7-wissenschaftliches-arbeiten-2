
% \mainmatter
\cleartoleftpage

\thispagestyle{empty}
\vspace*{\fill}
\newpage

\onehalfspacing
\pagenumbering{arabic}
\setcounter{page}{1}
\pagestyle{scrheadings}

\section{Quantisierung und Dither}

Eindimensionale Signale, die dem Shannon Abtasttheorem \autocite{shannon} unterliegen sind zeitlich diskret.
Aufgrund des Theorems kann die originale kontinuierliche Version des Signals ohne Informationsverlust rekonstruiert werden.
Die Wertemenge eines digitalen Signals wird, aufgrund endlichen Speichers, auch von einer theoretisch unendlich kontinuierlichen auf eine diskrete Wertemenge abgebildet.
Bei diesem Vorgang, der Quantisierung, gehen Informationen irreversibel verloren.
Ein solches diskretes Signal ist in \autoref{fig:quantized-signal} gezeigt.
In der digitalen Signalverarbeitung von Audiosignalen können sowohl \gls{floating point} als auch \gls{fixed point} Repräsentationen verwendet werden.
Der größte Unterschied dabei ist die dynamische Reichweite -- die Differenz zwischen dem Minimum und Maximum.
\autoref{tab:dynamic-range} führt die Reichweiten verschiedener Formate nach \citeauthor{dsp-guide} auf.
Aufgrund der enorm großen Dynamik und der Einfachheit der Implementation ist \gls{floating point} oft die Wahl für die Implementation für digitale Signalverarbeitung \autocite[S. 68ff]{dsp-guide}.

Zur Reproduktion von verarbeiteten Daten auf Endgeräten durch einen \gls{dac} oder auch nur zur endgültigen Speicherung oder Distribution wird eine \gls{fixed point} Repräsentation erfordert.
% \glsplural{dac} aktzeptieren generell keine Daten im \gls{floating point} Format.
% Onehin kann kein Gerät eine analoge Spannung oder Strom mit einer solchen Dynamik wiedergeben.
% Zur Speicherung und Distribution lohnt sich eine Effizientere Kodierung der Daten zu benutzten.
Daher ergibt sich die Notwendigkeit für Quantisierung.
Eine Quantisierungsfunktion $Q$ wird nach \citeauthor{dither} für eine beliebige Quantisierungsgröße $\Delta$ definiert als
\begin{equation}
    \setequationentry{\foreignquote{english}{Mid-tread} Quantisierungsfunktion}
    Q[n] = \Delta \left\lfloor \frac{n}{\Delta} + \frac{1}{2} \right\rfloor
\end{equation}

für einen arbiträren Abtastwert an Index $n$ \autocite[S. 500]{dither}.
Der Wert $\Delta$ repräsentiert dabei nach \citeauthor{dither} die Größe eines \gls{lsb}.
Der durch die Quantisierung entstehende absolute numerische Fehler ist die Differenz dessen mit dem Eingang
\begin{equation}
\setequationentry{Quantisierungsfehler}
E[n] = Q[n] - n
\end{equation}

Dieser Fehler ist periodisch und manifestiert sich als harmonische Verzerrung \autocites{dither}{noise-shaping}.
Diese ist in den meisten Anwendungsfällen eine unerwünschte Eigenschaft.
\marginline{Quantisierungsfehler können kreative Anwendung in der Musikproduktion finden.}

Aufgrund des Informationsverlusts durch die Quantisierung kann die Verzerrung des Signals nicht eliminiert werden.
Durch das Hinzufügen von nichtdeterministischem Rauschen -- dem Dither -- vor der Quantisierung kann die Periodizität der Quantisierung reduziert werden.
Das Signal bleibt jedoch verzerrt, nur nicht harmonisch.
Das Dither ist spektral Uniform und wird oft in einer \gls{tpdf} verwendet.
\gls{tpdf} bezieht sich auf die Form, die das Histogramm der Ditherwerte über einen längeren Zeitraum annimmt.
Ein \gls{tpdf} Dither setzt sich aus zwei unabhängigen Rauschquellen mit je einer Amplitude von $1$ \gls{lsb} zusammen \autocites{dither}[S. 23ff]{dsp-guide}.
% \marginline{Eine absolut positives Dither Signal darf nur subtraktiv gemischt werden um Überläufe zu vermeiden.}
Damit kann die harmonische Verzerrung des quantisierten Signals mit einem durch das Dither induzierten geringeren \gls{snr} \enquote{getauscht} werden.
Das Signal ist frei von unerwünschten Übertönen, jedoch nun mit einem stärkeren, spektral uniformem, Grundrauschen \autocites{dither}{noise-shaping}.
Für viele Anwendungsfälle ist dieser Ausgangszustand zufriedenstellend.

\begin{figure}[]
    \centering
    \begin{tikzpicture}
        \begin{axis}[
            axis lines=middle,
            domain=0:2,
            xlabel=$t$,
            ylabel=$A$,
            width=10cm,
            height=6cm,
            ymin=-1.1,
            ymax=1.1
        ]
        \pgfplotsinvokeforeach{1,0.75,...,-1}{
            \draw[dashed] (axis cs:0,#1) -- (axis cs:2,#1);
        };
        \pgfplotsinvokeforeach{0,0.08695652174,...,2}{
            \draw[loosely dotted] (axis cs:#1,-1.1) -- (axis cs:#1,1.1);
        };
        \addplot[
            color=THDStrongBlue,
            very thick,
            samples=100,
        ]{sin(deg(x * pi))};
        \addplot[
            color=THDGreen,
            samples=24,
            ycomb,
            mark=*,
        ]{round(4 * sin(deg(x * pi))) / 4};
        \addplot[
            color=THDDarkRed,
            thick,
            samples=24,
        ]{sin(deg(x * pi)) - (round(4 * sin(deg(x * pi))) / 4)};
        \end{axis}
    \end{tikzpicture}
    \caption[Quantisiertes Signal]{Quantisiertes Signal. Die grünen Datenpunkte weichen leicht von den originellen blauen ab. Die rote Linie zeigt den Quantisierungsfehler aus der Differenz zwischen dem kontinuierlichen und dem rekonstruierten diskreten Signal. Der Fehler kann maximal $\frac{1}{2}$\gls{lsb} betragen.}
    \label{fig:quantized-signal}
\end{figure}

\begin{table}[]
    \centering
    \caption[Dynamische Reichweite verschiedener Formate]{Dynamische Reichweite verschiedener Formate}
    \begin{tabular}{|l|c|c|c|}
        \hline
        \textbf{Format} & \textbf{Min} in \gls{dBFS} & \textbf{Max} in \gls{dBFS} & \textbf{Dynamik} in \gls{dBFS} \\ \hline \hline
        8-Bit Fixed & $-48$ & $0$ & $48$ \\ \hline
        16-Bit Fixed & $-96$ & $0$ & $96$ \\ \hline
        24-Bit Fixed & $-144$ & $0$ & $144$ \\ \hline
        32-Bit Fixed & $-192$ & $0$ & $192$ \\ \hline
        32-Bit Float & $-758$ & $770$ & $1528$ \\ \hline
    \end{tabular}
    \label{tab:dynamic-range}
\end{table}

\section{Rauschformung}

Ist ein uniformes Rauschen im Signal jedoch nicht zufriedenstellend kann dieses Rauschen auch spektral manipuliert werden.
In den folgenden Abschnitten wird das Konzept der Rauschformung und eine Methode zur numerischen Kreation eines Rauschformungsfilters erklärt.

\subsection{Ziele und Konzept}

Ein Rauschformungsfilter ist ein digitaler Filter der auf dem absoluten Fehler $e[n]$ der Quantisierung operiert und ein gefiltertes Signal rückkoppelt um ein gewünschtes Spektrum des Dithers zu erreichen.
Die schematische Darstellung dieses ist in \autoref{fig:noise-shaping} aus \autocite{noise-shaping} gezeigt.
\autoref{fig:noise-shaping-spectrum} zeigt ein erwünschtes Beispiel von Rauschformung im Spektralbereich.
Ein spektral uniformes Rauschen wird so manipuliert, sodass es nur minimal in einem bestimmten Frequenzbereich auftritt.
In Audioanwendungen liegt die generell anerkannte maximal reproduzierbare Frequenz bei $20\si{kHz}$.
Bei Abtastraten von nur dem doppelten (standardmäßig $44.1\si{kHz}$ oder $48\si{kHz}$) überdecken sich die Spektren des Rauschen mit dem Signal immer.
Bei identischer Bandbreite des Signals, jedoch höheren Abtastraten, wie $96\si{kHz}$ oder $192\si{kHz}$ kann mehr Quantisierungsrauschen in höhere, für das Signal irrelevante, Oktaven verschoben werden.
Dadurch wird der Effekt der Rauschformung verstärkt.
Hohe Abtastraten sind jedoch speicherplatz- und rechenintensiver und somit oft nicht praktikabel.

Das Ziel der Rauschformung ist es, einen Filter $H[z]$ zu konzipieren, um das Rauschen in einem System kontrolliert zu manipulieren.

\begin{figure}[t]
    \centering
    \begin{tikzpicture}
        \tikzstyle{block} = [
            draw,
            minimum width=2cm,
            minimum height=1.2cm
        ]

        \node[draw,
            circle,
            minimum size=0.6cm,
        ] (sum1) at (0,0){};

        \node[draw,
            circle,
            minimum size=0.6cm,
            right=2cm of sum1
        ] (sum2) {};

        \node [block,
            right=1cm of sum2
        ]  (Q) {$Q[n]$};

        \node [block,
            below=2cm of sum2
        ] (H) {$H[z]$};

        \node[draw,
            circle,
            minimum size=0.6cm,
            below=0.5cm of Q,
        ] (sum3) {};

        \draw (sum1.north east) -- (sum1.south west)
            (sum1.north west) -- (sum1.south east);

        \draw (sum2.north east) -- (sum2.south west)
        (sum2.north west) -- (sum2.south east);

        \draw (sum3.north east) -- (sum3.south west)
        (sum3.north west) -- (sum3.south east);

        \node[left=-1pt] at (sum1.center){\tiny $+$};
        \node[below] at (sum1.center){\tiny $-$};

        \node[left=-1pt] at (sum2.center){\tiny $+$};
        \node[above] at (sum2.center){\tiny $-$};

        \node[right=-1pt] at (sum3.center){\tiny $+$};
        \node[left=-1pt] at (sum3.center){\tiny $-$};

        \draw[->] (sum1.east) -- (sum2.west) node[midway, circ](xd){} node[midway, above]{$x' [n]$};
        \draw[<-] (sum2.north) -- ++(0,1) node[ocirc]{} node[above]{$d[n]$};

        \draw[->] (sum2.east) -- (Q.west);

        \draw[<-] (sum1.west) -- ++(-1,0) node[ocirc]{} node[left]{$x[n]$};
        \draw[->] (Q.east) -- ++(2,0) node[midway, circ](y){} node[ocirc]{} node[right]{$y[n]$};

        \draw[->] (xd) |- (sum3.west);
        \draw[->] (y) |- (sum3.east);
        \draw[->] (sum3.south) |- (H.east) node[above right]{$e[n]$};
        \draw[->] (H.west) node[above left]{$e' [n]$} -| (sum1.south);

    \end{tikzpicture}
    \caption[Blockdiagramm der Rauschformung]{Blockdiagramm der Rauschformung aus \autocite{noise-shaping}. $x[n]$ bildet den Eingang, $y[n]$ den Ausgang und $d[n]$ das Dither. $Q[n]$ stellt den Quantizer dar und $H[z]$ den Rauschformungsfilter, welcher hier als dessen Z-Transformation gezeigt ist.}
    \label{fig:noise-shaping}
\end{figure}

\begin{figure}[t]
    \centering
    \begin{subfigure}{6cm}
        \centering
        \begin{tikzpicture}
            \begin{axis}[
                xlabel=$f$,
                ylabel=$A$,
                ticks=none,
                width=6cm,
                xmin=0,
                xmax=1,
                ymin=0,
                ymax=1.2,
            ]
            \draw[pattern color=THDDarkRed, pattern=north east lines] (axis cs:0,0.4) -- (axis cs:0.52,0.4) -- (axis cs:0.6,0) -- (axis cs:0,0) -- cycle;
            \draw[very thick, THDDarkRed] (axis cs:0,0.4) -- (axis cs:1,0.4);
            \draw[very thick, THDGreen] (axis cs:0,1) -- (axis cs:0.4,1) -- (axis cs:0.6,0);
            \node[circle, fill=white, , inner sep=1pt] at (axis cs: 0.3, 0.2) {$A_1$};
            \node[circle, fill=white, inner sep=1pt] at (axis cs: 0.8, 0.2) {$B_1$};
            \node[anchor=south east] at (axis cs:0.4,1) {\textcolor{THDGreen}{Signal}};
            \node[anchor=south east] at (axis cs:0.4,0.4) {\textcolor{THDDarkRed}{Rauschen}};
            \end{axis}
        \end{tikzpicture}
        \caption{Ohne Rauschformung}
    \end{subfigure}
    %
    \textrightarrow
    %
    \begin{subfigure}{6cm}
        \centering
        \begin{tikzpicture}
            \begin{axis}[
                xlabel=$f$,
                ylabel=$A$,
                ticks=none,
                width=6cm,
                xmin=0,
                xmax=1,
                ymin=0,
                ymax=1.2,
            ]
            \draw[pattern color=THDDarkRed, smooth, pattern=north east lines] (axis cs:0,0.1) .. controls (axis cs:0.3,0.1) and (axis cs:0.4,0.28) .. (axis cs:0.53,0.36) -- (axis cs:0.6,0) -- (axis cs:0,0) -- (axis cs:0,0.1) -- cycle;
            \draw[very thick, THDDarkRed, smooth] (axis cs:0,0.1) .. controls (axis cs:0.5,0.1) and (axis cs:0.8,0.8) .. (axis cs:1,0.8) node[pos=0.7, above, sloped, text width=2cm, align=center] {\textcolor{THDDarkRed}{geformtes Rauschen}};
            \draw[very thick, THDGreen] (axis cs:0,1) -- (axis cs:0.4,1) -- (axis cs:0.6,0);
            \node[circle, fill=white, inner sep=1pt] at (axis cs: 0.45, 0.125) {$A_2$};
            \node[circle, fill=white, inner sep=1pt] at (axis cs: 0.8, 0.4) {$B_2$};
            \node[anchor=south east] at (axis cs:0.4,1) {\textcolor{THDGreen}{Signal}};
            \end{axis}
        \end{tikzpicture}
        \caption{Mit Rauschformung}
    \end{subfigure}
    \caption[Rauschspektren mit und ohne Rauschformung]{Rauschspektren mit und ohne Rauschformung. In Grün wird dar Spektralbereich des Signals gezeigt. Der rote Graph zeigt das Quantisierungsrauschen. Der schattierte Bereich zeigt die Schnittmenge des Rauschens im Signal. $A$ und $B$ sind die Flächen unter dem roten Graphen, es gilt $A_1+B_1 \equiv A_2 + B_2$. Jedoch ist $A_2 < A_1$.}
    \label{fig:noise-shaping-spectrum}
\end{figure}

\subsection{Arbiträre Rauschformungsfunktion}

Der Filter $H[z]$ ist ein herkömmlicher digitaler Filter.
Dieser könnte sowohl als \gls{fir} oder als \gls{iir} Filter implementiert werden.
Eine \gls{iir} Implementation ist jedoch aufgrund der nicht-konstanten Verzögerung und Phasenverschiebungen für diese Anwendung nicht zu nutzen \autocite{frequency-sampling-method}.
Da temporale Verzerrungen durch den Filter unerwünschte Interferenzen erzeugen könnten.
Der Filter wird somit als \gls{fir} in folgender Form aus \autocite{noise-shaping} definiert
\begin{equation}
\setequationentry{\acrshort{fir} Filter Übertragungsfunktion}
H[z] = \sum_{n=0}^{N-1} h[n] z^{-1}
\end{equation}
Dabei ist $N$ die Länge des Filters und $h[n]$ die Koeffizienten.
\marginline{Die Länge $N$ kann frei gewählt werden, sie sollte jedoch Teil von
$2^x \forall x \in \mathbb{N}_0^+$ sein.}

Wie in \autocite{noise-shaping} führen wir eine kontinuierliche Zielfunktion $W(\omega)$ ein.
$N$ sollte so gewählt sein, dass $W$ für eine bestimmte Nyquistfrequenz \autocite{shannon} nicht untertastet wird \autocite{SASPWEB2011}.
Die Funktion wird zuerst eindimensional interpoliert.
Dieser Vorgang ist abhängig von der jeweiligen Funktion.
Die arbiträre Funktion wird dann nach dem Häufigkeitsstichprobenverfahren numerisch zu Filter Koeffizienten umgewandelt.
Dabei wird die Funktion aus \autocite{frequency-sampling-method} zuerst diskretisiert mit
\begin{equation}
\setequationentry{Auswertung der Zielfunktion für das Häufigkeitsstichprobenverfahren}
W[k] = W \left( \frac{2\pi k}{N} \right) \quad k \in 0,1,2,\ldots,N-1
\end{equation}
% Da angenommen wird, dass
% \begin{equation*}
% W[k] \equiv H[k] \quad \text{gilt hier} \quad W[k] = H[k]
% \end{equation*}
Da der Kernel (die Koeffizienten) eines \gls{fir} Filters $h[n]$ per Definition die Antwort des Filters auf einen Dirac-Impuls $\delta[n]$ ist, können wir diese durch die inverse Fourier Transformation der Übertragungsfunktion von dem Bildbereich in den Zeitbereich gewinnen \autocites{frequency-sampling-method}{frequency-sampling-method-2}{SASPWEB2011}.
Es folgt, dass im kontinuierlichen Raum gilt
\begin{equation}
\setequationentry{Inversefouriertransformation im kontinuierlichen Zeitbereich}
h(t) = \frac{1}{2\pi} \int_{-\pi}^{\pi} W(\omega) e^{j t \omega} \text{d} \omega
\end{equation}
Diskret kann das angenähert werden durch
\begin{equation}
\setequationentry{Diskrete Inversefouriertransformation für \acrshort{fir} Koeffizienten}
h[n] = s \cdot \left| \textit{IDFT} \left( W[k] \right) \right| \cdot c[n]
\label{eq:coeffiecients}
\end{equation}
Der Faktor $s$ ist ein beliebiger Skalar um die Koeffizienten zu skalieren.
% Nach \autocite{frequency-sampling-method} könnte \autoref{eq:coeffiecients} auch explizit geschrieben werden als
% \begin{equation}
% h[n] = \frac{1}{2M+1} \left( W[0] + 2 \sum_{k=1}^{M} W[k] \cos \left( n \omega_k \right)  \right)
% \end{equation}
% mit $\omega_k = \frac{2\pi}{2N+1}k$ und $M = \frac{N-1}{2}$.
% Das wäre die inverse diskrete Fourier Tranformation.
Das Fenster $c[n]$ aus \autoref{eq:coeffiecients} kann frei gewählt werden, auch wenn manche bestimmten Fenster vor allem bei dieser Methode Nachteile abschwächen können \autocite{frequency-sampling-method-2}.
Dieses kann eine einfaches Kosinus Fenster sein, um die Trunkierungsartefakte zu minimieren
\begin{equation}
\setequationentry{Kosinusfensterfunktion}
c[n] = \frac{1}{2} \cos \left( \frac{2\pi \left( x - \frac{N}{2}\right) }{N}\right) + \frac{1}{2}
\end{equation}

\subsection{Psychoakustisch Ideale Funktion}

Als psychoakustische Funktion kann die nicht-linearität des menschlichen Ohrs ausgenutzt werden.
Eine ideale Funktion kann aus einer modifizierten Abwandlung der $0$ Phon \gls{ath} Kurve aus \autocite{iso226} gewonnen werden.
\marginline{\enquote{Phon} ist eine absolute logarithmische Einheit zur Messung von akustischer Lautstärke definiert in \textit{ISO 532}.}
Diese kann numerisch generiert werden.
Anschließend wird diese durch Interpolation auf die gesamte relevante Bandbreite extrapoliert und in der Amplitude normalisiert werden.
Wir erhalten die Funktion $A(\omega)$.
Es kann nun durch Hartes oder Soft-Clipping ein Maximum der Funktion festgelegt werden.
Beispielsweise kann
\begin{equation}
\setequationentry{Soft-Clipping Sättigungsfunktion}
S(m, \omega) = m \tanh \left( \frac{A(\omega)}{m} \right)
\end{equation}
als Soft-Clipping Funktion mit dem Maximum $m$ verwendet werden.
Durch das numerische Lösen eines \foreignquote{english}{Least-Squares} Problems, wie in \citeauthor{noise-shaping}, kann ein optimaler Gain gefunden werden.
Dafür muss das globale Minimum von dem Gain $g$ für
\begin{equation}
\setequationentry{Gainoptimierung der Zielfunktion}
f(g, m) = \left( \frac{1}{\pi} \int_{0}^{\pi} S(m, \omega) + g \text{d}\omega \right)^2 = 0
\end{equation}
gefunden werden \autocite{noise-shaping}.

\section{Diskussion}

Die konstruierte Filterfunktion wird nochmal in \autoref{fig:frequency} im Vergleich zur psychoakustischen Referenz $W[k]$ illustriert.
Die Methode ist nicht garantiert akkurat zu sein, jedoch kann sie relativ effizient berechnet werden.
Andere Referenzfunktionen, als die hier präsentierte, könnten bessere Ergebnisse liefern.
Diese Methode ist auch nur Offline ausführbar; ein Online Version für temporale Rauschformung ist nicht vorgesehen, könnte jedoch möglicherweise auf leistungsstarken parallelen System realisiert werden.
Dazu müsste $W[k]$ in einer Abhängigkeit von $x[n]$ oder nur $n$ bzw. $t$ definiert werden.
Im Vergleich zu \autocite{noise-shaping} liefert diese Methode vergleichbare Resultate.
Der Unterschied liegt jedoch darin, dass hierbei arbiträre Funktionen verwendet werden können.
Die \gls{ath} Kurve von \autocite{iso226} wurde nur als Beispiel verwendet;
arbiträre Kurven können auch für andere Anwendungsgebiete außerhalb von Audio, welche von Rauschformung profitieren, durch diese Methode konstruiert werden.

Um ideale Resultate des Quantisierungsprozesses zu erhalten, muss anwendungsspezifische Erfahrung in den Entscheidungsprozess mit einfließen.
Eine pauschale Aussage über die spezifische Ausführung oder gar Notwendigkeit von Rauschformung für eine gegebene Applikation kann nicht gegeben werden und ist offen zu weiterer Forschung.

\begin{figure}[h]
    \centering
    \begin{tikzpicture}
        \begin{axis}[
            xlabel=$f$ in Hz,
            ylabel=Relative Amplitude,
            width=\linewidth,
            height=7cm,
            xmode=log,
            log ticks with fixed point,
            yticklabels={,,},
            xmin=20,
            xmax=30000,
            grid=both,
            legend pos=south west,
        ]
        \addplot[color=THDStrongBlue,smooth,thick] table[col sep=comma,header=true,x index=1,y index=2] {./data/ath.csv};
        \addlegendentry[]{\gls{ath} nach \texttt{ISO226}};

        \addplot[color=THDGreen,smooth,thick] table[col sep=comma,header=true,x index=1,y index=2] {./data/fir.csv};
        \addlegendentry[]{\gls{fir} Filter};

        \end{axis}
    \end{tikzpicture}
    \caption[Normalisierte Bodeplot des Filters]{Normalisierte Bodeplot des \gls{fir} Filters und der modifizierten \gls{ath}. Der \gls{fir} Filter wurde mit 512 Taps für die Frequenzen von $20\text{Hz}$ bis $24\text{kHz}$ (äquivalent zu $\frac{f_s}{2}$) simuliert.}
    \label{fig:frequency}
\end{figure}
