{
\onehalfspacing
\centersection{Abstrakt}

\begingroup
\leftskip4em
\rightskip\leftskip

Im letzten Schritt einer digitalen Audioverarbeitungs-Pipeline wird ein Signal mit hoher Bit-Tiefe, meistens im \gls{floating point} Format, in eine niedrigere Bit-Tiefe konvertiert um das Signal praktisch verwenden zu können.
Bei dieses Quantisierungsprozesses entstehen Quantisierungsfehler.
Diese Verzerrungen können durch Dither gemindert und anschließend durch Rauschformung spektral manipuliert werden.
In dieser Arbeit wird eine numerisches Methode anhand des Häufigkeitsstichprobenverfahrens zur Ermittlung einer arbiträren Rauschformungsfunktion präsentiert.
Dieses wird dann zur Erstellung eines psychoakustischen Rauschformungsfilters basierend auf \textit{ISO226} angewendet.
Diese Methode liefert offline berechnete Koeffizienten für einen \gls{fir} Filter.

\endgroup
}
