{
\onehalfspacing
\centersection{Abstrakt}

\begingroup
\leftskip4em
\rightskip\leftskip

Im letzten Schritt einer digitalen Audioverarbeitungs-Pipeline wird das Signal mit hoher Bit-Tiefe, meistens \gls{floating point}, in eine niedrigere Bit-Tiefe konvertiert um das Signal praktisch verwenden zu können.
Bei diesem Quantisierungsprozess entstehen Quantisierungsfehler.
Diese Verzerrungen können durch Dither gemindert und anschließend durch Rauschformung spektral manipuliert werden.
In dieser Arbeit wird ein numerisches Verfahren anhand des Häufigkeitsstichprobenverfahrens zur Ermittlung einer arbiträren Rauschformungsfunktion präsentiert.
Dieses wird dann zur Erstellung eines psychoakustischen Rauschformungsfilters basierend auf \textit{ISO226} angewendet.
Diese Methode liefert rechen-effiziente Filterkoeffizienten, die statisch, offline, errechnet werden können.

\endgroup
}
