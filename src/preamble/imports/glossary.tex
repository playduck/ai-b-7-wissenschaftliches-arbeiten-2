% Robin Prillwitz 2022

\usepackage[
  acronym,
  toc,
  nonumberlist
]{glossaries}
\usepackage[automake, nonumberlist, nogroupskip]{glossaries-extra}

\renewcommand*{\glsclearpage}{} % remove gloassary pagebreak
\setabbreviationstyle[acronym]{long-short}
% \renewcommand*{\glstextformat}[1]{\textit{#1}} % cursive acronyms
\setglossarysection{subsection}

\setglossarystyle{long}
\renewenvironment{theglossary}%
 {\begin{longtable}[l]{lp{\glsdescwidth}}}%
 {\end{longtable}}
\renewcommand{\glsnamefont}[1]{\textbf{#1}}

% --------------------------------- Acronyms -------------------------------- %
\newacronym{snr}{SNR}{Signal to Noise Ratio}
% \newacronym{fpu}{FPU}{\gls{floating point} Unit}
% \newacronym{fft}{FFT}{Fast Fourier Transform}
\newacronym{dac}{DAC}{Digital Analog Converter}
\newacronym{lsb}{LSB}{Least Significant Bit}
\newacronym{tpdf}{TPDF}{Triangular Probability Density Funtion}
\newacronym{fir}{FIR}{Finite Impulse Response}
\newacronym{iir}{IIR}{Infinite Impulse Response}
\newacronym{ath}{ATH}{Absolut Threshhold of Hearing}

% --------------------------------- Glossary -------------------------------- %
\newglossaryentry{floating point}{name={Floating Point}, description={Gleitkommazahl, oft nach IEEE754 implementiert}}
\newglossaryentry{fixed point}{name={Fixed Point}, description={Ganzzahl}}
\newglossaryentry{dBFS}{name={dBFS}, description={Dezibel Full-Scale. Der Maximale Wert, den ein digitales System verarbeiten kann liegt bei $0$\gls{dBFS}}}
\newglossaryentry{soft-clipping}{name={Soft-Clipping}, description={Weiche Begrenzung an einen Maximalwert}}
\newglossaryentry{hard-clipping}{name={Hard-Clipping}, description={Harte, nichtlineare Begrenzung}}
\newglossaryentry{phon}{name={Phon}, description={ Logarithmischer Lautstärkepegel, drückt die empfundene Lautstärke aus (Auch bekannt als Fon)}}

\glsenablehyper
\makeglossaries
