% Robin Prillwitz 2022
{
\thispagestyle{empty}
\vspace*{\fill} % Position at the bottom
\def \colophonwidth {8cm}

\vspace*{0.5em}
\singlespacing
\parskip=0pt
\setstretch{1}\noindent

{\hsize\colophonwidth\hfill\vbox{\parskip=0pt\parfillskip0pt%
\emergencystretch\colophonwidth\noindent{

    \centersection{Kolophon}

    \begin{center}
        \begin{tikzpicture}
        \pgfplotsset{colormap={}{color=(THDWhite) color=(THDGrey) }}
        \begin{axis}[
            width=10cm,
            hide axis,
            enlargelimits=false,
            disabledatascaling,
            % view={23.5}{30},
            view={20}{50},
            % domain=-2:2,
            % domain y =-2:2,
            shader=interp,
            % z buffer=sort,
        ]
        \begin{scope}[blend group=darken]
            % \addplot3[mesh, samples=200]{(sin(deg(x*pi))/(x*pi)) * (sin(deg(y*pi))/(y*pi))};
            \addplot3[
                mesh,
                % opacity=0.5,
                domain=-1.5:1.5,
                y domain=0:360,
                samples=100,
                samples y=50,
                ]
            (
                {x*cos(y)}, {x*sin(y)},
                {(sin(deg(x*cos(y)*pi))/(x*cos(y)*pi)) * (sin(deg(x*sin(y)*pi))/(x*sin(y)*pi))}
            );
        \end{scope}
        \end{axis}
     \end{tikzpicture}
     \textcolor{THDGrey!40!THDWhite}{
     \footnotesize{
         $$\text{sinc}\left(x,y\right) = \frac{\sin \left( \pi x \right) \sin \left( \pi y\right)}{\pi^2 x y}$$
         \textit{Sinus Cardinalis}
    }}
    \end{center}

    \noindent
    % \begin{tikzpicture}
    %     \node[] (A) at (0,0) {};
    %     \node[] (B) at ($(\colophonwidth - 1.4ex,0)$) {};
    %     \pgfornamentline[color=THDBrightRed]{A}{B}{1}{88}
    % \end{tikzpicture}

    \vspace*{6pt}

        Diese Arbeit ist ein \hologo{LaTeX3} Dokument in \textit{Markus Kohm}'s \hologo{KOMAScript} Klasse und wurde in \textit{Minion Pro} 11pt gesetzt.
        Die Typographie nimmt Inspiration von \textit{Robert Bringhurst} und \textit{Edward Tufte}.
        Alle Darstellungen sind Eigenkreationen erstellt mit Ti\textit{k}Z, \textsc{pgfplots} und \textsf{Matplotlib}.
        Kompiliert wurde das Dokument mit \hologo{LuaLaTeX} und \hologo{biber} auf \platformname\ am \today.

    \begin{center}
        \HighlightColor{\textsc{
            \authorFirst\;\authorLast
        }}\\
        \makeatletter
        \HighlightColor{\textbf{
            \@Roman{\the\year}.
            }}
        \makeatother
    \end{center}

    % \color{THDBrightRed}{
    %     \noindent\hfill\pgfornament[width=20pt]{1}\hfill\null\par
    % }

}}\hfill}

}
